
\documentclass[a4paper,12pt]{article}

\usepackage{../acn-scientific}

%\title{AQA Unit PHYA1\\Particles, Quantum Phenomena and Electricity}
\title{AQA Unit PHYA1\\Electromagnetic Radiation and Quantum Phenomena}
\author{Mr \textsc{A.C. Norman}\\
\textsc{Bishop Heber High School}
\\ \texttt{anorman@bishopheber.cheshire.sch.uk}}
\date{Autumn Term, 2011}

\begin{document}
\begin{titlepage}
\maketitle

\thispagestyle{empty}
\enlargethispage{4cm}
	
\begin{center}
	%\includegraphics[width=0.8\textwidth]{circuit_diagram.png} 
\end{center}

\begin{flushright}
%\texttt{http://xkcd.com/730/}
\end{flushright}
\end{titlepage}

%\tableofcontents
\clearpage
%\part{Quantum Phenomena}

\section{The photoelectric effect}
Work function $\phi$, threshold frequency $f_{0}$, photoelectric equation $hf=\phi+E_{k}$, stopping potential experiment not required

\section{Collisions of electrons with atoms}
The eV
Ionization and excitation, understanding of ionization and excitation in a fluorescent tube



\section{Energy levels and photon emission}


Line spectra (e.g. of atomic hydrogen) as evidence of discrete transitions between energy levels in atoms
$hf=E_{1}-E_{2}$

\section{Wave-particle duality}
electron diffraction suggests the wave nature of particles and the photoelectric effect suggests the particle nature of EM waves.  details of particular diffraction methods not required.%WHat is diffraction? How can we do this prior to Waves module?
de Broglie wavelength $\lambda=\frac{h}{mv}$
where $mv$ is the momentum

\end{document}
