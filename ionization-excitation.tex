
\section{Collisions of electrons with atoms}

When electrons collide with atoms, what happens depends on the speed (and therefore the kinetic energy) of the electron.  At very low energies, the electron will collide elastically with the atoms, i.e.\ they will bounce off without losing any energy.  As the energy is increased, inelastic collisions start to occur, and the electrons stand a chance of losing energy to the atom, changing its internal arrangement in the process.

\subsection{Ionization}

An ion is a charged atom, one which has gained or lost one (or more) electrons, so that the number of electrons no longer equals the number of protons, as it does in a neutral atom.  We have already seen that in a metal, most of the atoms have at least one electron which can move freely around the metal, leaving them as positive ions.  Atoms can become ionized by gaining or losing electrons in lots of different ways, including chemical bonding (and in particular ionic bonding); having their electrons knocked out by passing alpha or beta particles from radioactive substances; decaying radioactively themselves by $\alpha$ or $\beta$ decay; having their electrons annihilated by positrons; or being subjected to a too strong electric or magnetic fields.

A \emph{plasma} is any ionized gas: a gas made up partly or completely of charged particles like ions\footnote{Most astrophysicists think that 99.9\% of all observable matter is in the form of plasma.}.  We most often encounter such a gas when electrons pass through a gas, with enough energy to knock electrons out of the gas atoms and make them into ions.

The energies required are often measured using the unit of the electron volt (\si{eV}).\footnote{An electron volt (\si{eV}) is equal to the energy an electron has gained when it has moved through a potential difference of one volt.  Since the work done when a particle of charge $Q$ moves through a potential $V$ is $E=QV$, an electron which has a charge of \SI{1.6e-19}{C} will gain \SI{1.6e-19}{J} of (kinetic) energy when it is accelerated through a p.d. of \SI{1}{V}.}  At first this may seem an odd unit of energy to use, compared to the \si{J} with which you are more familiar, but it is just the right size and is therefore most useful when dealing with energies on an atomic scale.

\[\SI{1}{eV}=\SI{1.6e-19}{J}.\]

The minimum amount of energy that is needed to ionize an atom (i.e.\ to remove its most loosely bound electron) is known as the \emph{ionization energy}.  For example, the ionization energy of hydrogen is \SI{13.6}{eV} (or \SI{2.18e-18}{J}).  Of course, hydrogen only has one electron, but for other elements, we could go on removing electrons.  The energy needed to remove the next most loosely bound electron is called the second ionization energy, and so on\ldots

We can see from the definition of the electron volt that \SI{13.6}{eV} is the kinetic energy gained by an electron when it is accelerated through a p.d. of \SI{13.6}{V}.  Therefore, if an electron which has been accelerated from rest through a p.d. of \SI{13.6}{eV} or more collides with a hydrogen atom, it has enough energy to knock the hydrogen's electron out of the atom and leave the hydrogen ionized.  This in fact is a common way to produce ionized gases in the laboratory, via a low pressure gas between electrodes in a long narrow glass tube, and such an arrangement is known as a discharge tube.  The ionized gas glows a characteristic colour, for reasons we shall discover when we discuss line spectra. %section \ref{photon-emission}.

\subsection{Excitation}

Gas atoms can still absorb energy from electrons without being ionized.  In fact, even if the incoming electron does not have enough energy to completely remove an electron (and the easiest electron to remove is the most loosely bound one), it might be able to `excite' the atom by giving one of its electrons some energy.  The excited electron can then move into a higher energy state (one which is further from the nucleus).

For example, the first and second \emph{excitation energies} of hydrogen are \SI{10.2}{eV} and \SI{12.1}{eV} respectively.  An energetic incoming electron striking a hydrogen atom might bounce off elastically, losing no energy; it might excite the hydrogen atom, losing one of these amounts of energy doing so; or it might ionize the hydrogen atom completely, losing \SI{13.6}{eV} of energy.  There will be a certain probability of each occurring.
%check this Calculation of the 1s-2s Electron Excitation Cross Section of Hydrogen 1958 Proc. Phys. Soc. 72 121 Excitation by electron collision of excited atomic hydrogen K Omidvar - Physical Review, 1965 - APS J. Phys. Chem. Ref. Data 19, 617 (1990); http://dx.doi.org/10.1063/1.555856 (20 pages) Cross Sections and Related Data for Electron Collisions with Hydrogen Molecules and Molecular Ions
