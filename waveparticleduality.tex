
\section{Wave-particle duality}
We have seen some phenomena, e.g.\ refraction, which can be explained by thinking of light behaving as though it were a wave.  Other phenomena, such as the photoelectric effect, suggest that light comprises particles.\footnote{The human eye is a very good instrument: it takes only five or six photons to activate a nerve cell and send a message to the brain.  If we were evolved a little further so we could see ten times more sensitively, we shouldn't have to have this discussion: we should all have seen very dim light of one colour as a series of intermittent flashes of equal intensity, as the individual photons hit our retina.}  In fact it is useful in some circumstances to describe light---and here we don't mean just the light we can see, but all sections of the electromagnetic spectrum---as a wave and in others look upon it as consisting of particles; these are both useful models that can be applied to different situations.  We find a strong analogy here to the fable of the seven blind men who ran into an elephant.  One man felt the trunk and said ``the elephant is a rope''; another felt the leg and said ``the elephant is a tree,'' and so on.

In 1924, Louis de Broglie (1892--1987) suggested that if light (being normally though of as a wave) can be thought of as particle, then things which we usually consider to be particles may have wavelike properties.  If a `particle' acts like a `wave' then it must have an associated wavelength.  This wavelength, de Broglie postulated, would be related to the momentum $p$ of a particle by
\[\lambda=\frac{h}{p}=\frac{h}{mv},\]
where $h$ is Planck's constant, and the wavelength $\lambda$ became known as the \emph{de Broglie wavelength}.

\subsection{Evidence supporting de Broglie's hypothesis}
The first evidence in support of this came in 1927 when electron diffraction was observed by two separate teams of scientists, George Paget Thomson (who passed a beam of electrons through a thin metal film) at the University of Aberdeen, and Clinton Davisson and Lester Germer (who sent an electron beam through a crystalline grid) at Bell Labs in the US.  This led to de Broglie being awarded the Nobel Prize for Physics in 1929 for his hypothesis. Thomson and Davisson shared the Nobel Prize for Physics in 1937 for their experimental work.

Davisson and Germer diffracted electrons from the surface of a nickel crystal.  They accelerated electrons through a high voltage, and fired them at the crystal observing the reflected electrons. They observed a diffraction pattern as the planes of the crystal act like a diffraction grating.  In their experiment, Davisson and Germer used \SI{5000}{V} to accelerate the electrons, giving a de Broglie wavelength of \SI{1.7e-11}{m} (this is equivalent to an X-ray wavelength for light, so the electrons should behave similarly to X-rays).\footnote{The electrons are given their kinetic energy by the accelerating voltage, so $\frac{1}{2}m_{e}v^{2}=eV$ or, rearranging, $mv=\sqrt{2m_{e}eV}$.  This allows us to determine the de Broglie wavelength $\lambda=\frac{h}{mv}=\frac{h}{\sqrt{2m_{e}eV}}=\SI{1.7e-11}{m}$.}  Since this wavelength is approximately equal to the crystal plane spacing, diffraction occurs.

Since these early particle diffraction experiments, protons, neutrons, and hydrogen and helium atoms have been diffracted and thus shown to have wavelike properties.  Larger everyday objects (often termed `macroscopic') will not undergo diffraction as their wavelength turns out to be smaller that any possible diffraction setup (e.g.\ a snooker ball moving at \SI{1}{m.s^{-1}} has a wavelength of approximately \SI{e-33}{m}).

%\subsection{Double slit interference with electrons}

%If we were to perform a double slits type experiment using electrons rather than light, we would get a pattern very similar to that produced with light, with the fringes being thousands of time closer together.

%It is obvious to think that half of the electrons pass through S1 and half through S2 and that the interference between the two sets of electrons causes the pattern.  But, this is NOT what happens.
%	If we reduce the intensity of the electron beam, so that only one electron arrives at the slits at any one time, we still get the interference pattern, as long as the time is made long enough.  The diagram below shows what the film looks like as the pattern builds up. 


%What must happen is that  each electron passes through both slits and interferes with itself.  It is not possible to predict where an electron will go after passing through the slits, but only to give probabilities - it is most likely to go to a light area, and least likely to go to a dark area.
%	The idea of an electron passing through both slits at the same time is clearly not consistent with the idea of a particle, but is understandable if we accept that electrons act like a wave. However, if we watch the electrons passing through the slits, so we know which slit an electron has gone through (and therefore make it act like a particle), the interference pattern disappears. 

%	Wave particle duality means that neither the particle or wave model can be used to explain everything that light or matter does.  On a small scale, these models will only work in certain circumstances.  The best model available is quantum mechanics, which describes in abstract mathematics what is happening, and abandons any relation to a "physical" picture.

%Electrons are behaving like waves here.  This is an example of the so-called `wave--particle duality'.


